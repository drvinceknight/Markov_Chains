\documentclass[12pt]{article}

%packages
\usepackage{graphicx}
\usepackage{amsmath}
\usepackage{mathdots}
\usepackage{amsthm}
\usepackage{amssymb}
\usepackage[dvips]{color}
\usepackage{fancyhdr}
\usepackage{pstricks}
\usepackage{pst-node}
\pagenumbering{arabic}
\usepackage{hyperref}
\usepackage{lscape}
%Margins etc...
\setlength{\textheight}{240mm}
\setlength{\topmargin}{-17mm} \setlength{\oddsidemargin}{-4mm}
\setlength{\textwidth}{166mm} \setlength{\parindent}{0mm}
\setlength{\marginparsep}{9mm} \setlength{\parskip}{3mm}

\begin{document}
\begin{center}
\Huge{Markov Chains Exercise Sheet}\\
\date{\tiny{Last updated: \today.}}
\end{center}

\begin{enumerate}
\item Assume that a student can be in 1 of 4 states:
\begin{itemize}
	\item Rich
	\item Average
	\item Poor
	\item In Debt
\end{itemize}
Assume the following transition probabilities:

\begin{itemize}
	\item If a student is Rich, in the next time step the student will be:
	\begin{itemize}
		\item Average: .75
		\item Poor: .2
		\item In Debt: .05
	\end{itemize}
	\item If a student is Average, in the next time step the student will be:
	\begin{itemize}
		\item Rich: .05
		\item Average: .2
		\item In Debt: .45
	\end{itemize}
	\item If a student is Poor, in the next time step the student will be:
	\begin{itemize}
		\item Average: .4
		\item Poor: .3
		\item In Debt: .2
	\end{itemize}
	\item If a student is In Debt, in the next time step the student will be:
	\begin{itemize}
		\item Average: .15
		\item Poor: .3
		\item In Debt: .55
	\end{itemize}
\end{itemize}
\begin{enumerate}
\item Draw the corresponding Markov chain and obtain it's stochastic matrix.
\item Let us assume that a student starts their studies as ``Average''. What will be the probability of them being ``Rich'' after 1,2,3 time steps?
\item What is the steady state probability vector associated with this Markov chain?
\end{enumerate}
\item Consider the following matrices, are they stochastic matrices:

For the matrices that are stochastic matrices, obtain the steady state probabilities (if they exist, if not, explain why).
\end{enumerate}

\end{document}
