\documentclass[12pt]{article}

%packages
\usepackage{graphicx}
\usepackage{amsmath}
\usepackage{mathdots}
\usepackage{amsthm}
\usepackage{amssymb}
\usepackage{fancyhdr}
\usepackage{pstricks}
\usepackage{pst-node}
\pagenumbering{arabic}
\usepackage{hyperref}
\usepackage{lscape}
%Margins etc...
\setlength{\textheight}{240mm}
\setlength{\topmargin}{-17mm} \setlength{\oddsidemargin}{-4mm}
\setlength{\textwidth}{166mm} \setlength{\parindent}{0mm}
\setlength{\marginparsep}{9mm} \setlength{\parskip}{3mm}

\begin{document}
\begin{center}
\Huge{Markov Chains Sage Sheet}\\
\tiny{Last updated: \today.}
\end{center}

\begin{enumerate}
\item Discrete Markov Chains\\
Navigate to the Sage code snippet at: \url{http://interact.sagemath.org/node/39} and perhaps start a new Sage worksheet so that you can try a few of your own Sage commands.

WARNING: The convention used in this code snippet is different to the notes. The columns sums add to 1 not the row sums. Keep this in mind as you work through it.

\begin{enumerate}
\item Using the default inputs, what is the steady state distribution associated with this Markov chain (try and use the Sage ``solve" command to verify this)?
\item How long does it seem to take to arrive at that state (try and use Sage to verify this)?
\end{enumerate}
\item Continuous Markov Chains\\
The evolution of a Continuous Markov Chain obeys the following differential equation:

$${d\pi(t)\over dt}=\pi(t)Q$$
which has solution:
$$\pi(t)=\pi(0)e^{Qt}$$
This requires the calculation of the exponential of a Matrix. This is not straightforward! One approach is to use the following formula:
$$
e^{M}=\mathbb{I}+\sum_{k=1}^{\infty}{M^k\over k!}
$$
Navigate to the Sage code snippet at \url{http://interact.sagemath.org/node/71} and experiment with the code to see how good this approach is.
\begin{enumerate}
\item Consider the following Markov Chain:
$$Q=\begin{pmatrix}
-3&3&0&0\\
4&-7&3&0\\
0&4&-7&3\\
0&0&4&-4
\end{pmatrix}$$
\item Use the Sage ``solve" command to obtain the steady state probabilities for this Markov Chain.
\end{enumerate}
\end{enumerate}

\end{document}
